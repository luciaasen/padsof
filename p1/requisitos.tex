
\documentclass{article}

\usepackage{lmodern}
\usepackage[T1]{fontenc}
\usepackage[utf8]{inputenc}
\usepackage{mathtools}
\usepackage{graphicx}
\newcommand\tab[1][0.6cm]{\hspace*{#1}}
\renewcommand\labelenumii{\theenumi.\arabic{enumii}.}
\renewcommand\labelenumiii{\labelenumii\arabic{enumiii}.}
\renewcommand\labelenumiv{\labelenumiii\arabic{enumiv}.}

\renewcommand\labelitemi{$\cdot$ }
\title{Requirements Analysis Document}
\author{Lucía Asencio y Juan Riera}

\begin{document}
\maketitle
\begin{enumerate}
	\item{Introduction}
	\begin{enumerate}
		\item{Purpose of the system}
		\\The system is an application with educational purposes. It offers a communication and evaluation platform between professors and students.
		\item{Scope of the system}
		\\The application should have two types of users: students and teachers. There will be courses, created by teachers, which will be divided in units, subunits, subsubunits and so on. In each unit there will be exercises and notes all of which are organized by subjects.. Therefore it also serves as a database, that also provides a statistic calculation system detailed below.\\Students will have as many accounts as te number of students using the application, however, there will only be one teacher account for all teachers.
		\item{Objectives and success criteria of the project}
		\\The objective is to create a user-friendly application that meets the functionality detailed in the next section.
		\item{Definitions, Acronyms and abbreviations}
		\\Although sometimes in this document we will talk about units and sometimes about courses, we will be referring in both cases to the same thing.
	\end{enumerate}
	
	\item{System Description}
	\begin{enumerate}
		\item{Functional Requirements}
		The application has two types of users:
		\begin{enumerate}
			\item User type 1: student
			\begin{enumerate}
				\item Log in to the application\\
				\item View available courses and send an application to a course and get e-mail notifications when they are accepted or declined.\\
				\item Access to each subject unit/subunit/subsub... and unit notes\\
				Each item will only be visible after certain date (decided by teacher) if the student belongs to the course and is not expelled.\\
				\item Access and solve exercises, each of which belongs to a unit, which belongs to a course. Exercises will only be visible and solvable after and before certain dates decided by teacher \\
				A student may leave a question unanswered, and may quit the exercise without sending it whenever he wants, his answers will not be saved.\\
				Once a student finishes an exercise, he cannot do it again. \\
				Each exercise will have a relevance specified by the teacher, this relevance should be displayed at any time.
				After the deadline of an exercise, students can view their marks in that exercise (normalized from 0 to 10), as well as correct answers and the answers they wrote.\\
				\item See their mark in a course up to that point, normalized from 0 to 10 \\
				\item Communicate with teachers and other students via e-mail		\\
				
			\end{enumerate}
			\item User type 2: teacher
			\begin{enumerate}
				\item Log in to the application\\
				\item Get notifications via email when a student applies to a course.\\
				\item Accept/decline students applications to a certain course\\
				\item Set up courses, unit, subunits, subsubunits...\\
				Also, write and add  notes to each unit, subunit etc. This notes will be plain text.\\
				\item Use an exercise maker to add exercises to units\\
				This maker will set the unit where the exercise belongs, the two dates between which it is doable, the number of questions and the type of each of them (multiple choice/one choice/true-false/open answer), the order of the questions (which can be chosen to be random or a fixed order, also chosen by the teacher), the penalty for each wrong answer (if any), the value of the exercise for the final course mark (all of which don't have to sum 10, since they will be normalized), the text of the question, the points of each question, the correct answer and the possible answers in the multiple choice/one choice type of questions\\
				Teacher can re-set up the whole exercise (excepting for the dates: only deadline can be changed and only if it is postponed), if no student has already done the exercise. (Students will get a notification whenever the exercise turns visible)\\
				\item Make notes and exercises (and, optionally, courses) visible or invisible. This visibility may also be scheduled.\\
				\item Access to all the information of a student, except for his log in data and student's emails. This means teachers have access to every mark of every student, in every exercise and every course, every answer he has given to every question in every exercise he has finished. He also will have access to the course average marks of every student.\\
				\item Access courses list.\\
				\item View statistics\\
				Statistics per subject : all the normalized marks, average mark, number of fails and number of passes.\\
				Statistics per exercise : marks, average mark for each exercise.\\
				Statistics per question : not answered, answered, correct answers, wrong answers.\\
				\item Expel students (and readmit them in the course), also access to a list of the expelled students\\
				\item Communicate with students via e-mail\\
				
				 
			\end{enumerate}
		\end{enumerate}
		\item{Non-functional Requirements}
			\begin{enumerate}
				\item The application should be prepared to be used by the common user who does not have computer knowledge further than opening google.
				\item The statistics calculations should not take longer than a few seconds.
			\end{enumerate}
	\end{enumerate}
	\newpage
	\item{Use Cases}
	\begin{enumerate}
		\item{Use Case diagram}
		\\
		We divided our diagram in three parts so that it was easily understandable.\\
		\begin{itemize}
			\renewcommand\labelitemi{$\star$}
			\item First one is for student use cases
			\item Second one is for teacher use cases
			\item Third one is also for teacher use cases \\ \\
		\end{itemize}
	
		\includegraphics[width=125mm]{Diagram_1.png}
		\includegraphics[width=125mm]{teacher1.png}
		\includegraphics[width=125mm]{teacher2.png}
		\\
		\item{Use case descriptions}
		\begin{enumerate}
			\item{Use case: View statistics per subject}
			\begin{itemize}
				\item Primary Actor: Teachers
				\item Stakeholders and Goals: Teachers who want to check general statistics (this means, not student specific) 
				\item Preconditions: the user has logged in as a teacher, accessed the courses list and selected one.
				\item Success Guarantee: the needed statistics are displayed in some seconds.
				\item Main Success Scenario: 
					\begin{itemize}
					\item 1. The teacher selects "Course statistics and data"
					\item 2. The following data are displayed:
						\begin{itemize}
							\item Average mark in the subject.
							\item Number of passes in the subject.
							\item Number of fails in the subject.
							\item List of all the student's normalized marks in the subject.
							\item List of exercises in the subject.
						\end{itemize}
				
					\end{itemize}
				\item Extensions: 
					\begin{itemize}
						\item 2a. The user clicks on an exercise.
						\item 3a. The following data is displayed:
							\begin{itemize}
								\item Average mark in the exercise.
								\item Number of passes in the exercise.
								\item Number of fails in the exercise.
								\item Relevance of the exercise.
								\item List of questions.
							\end{itemize}
						The teacher may also click on one question from the list: \\
							\item 3b. The teacher clicks on one question from the list displayed in 3a.
							\item 4b. The following data is displayed:
								\begin{itemize}
									\item Number of correct answers.
									\item Number of incorrect answers.
									\item Number of students who did not answer.
									\item Correct answer.
									\item List of answers given by the students.
								\end{itemize}
					\end{itemize}
				\item Special Requirements: Quick response (a few seconds) in the process of calculating and displaying the data.
				\item Frequency: no concurrency, one user at a time.
			\end{itemize}
		
		
			\item{Use case: Teacher makes an exercise}
						
			\begin{itemize}
				\item Primary Actor: Teachers
				\item Stakeholders and Goals: Teachers who want to use the exercise maker so that they can evaluate the students
				\item Preconditions: Have logged in as a teacher
				\item Success Guarantee: The exercise will be added to the course exercises.
				\item Main Success Scenario:
				\begin{itemize}
					\item 1. Teacher selects "Add exercise".
					\item 2. A list of available courses is displayed.
					\item 3. Teacher selects the course where to add the exercise.
					\item 4. A list of units is displayed.
					\item 5. Teacher selects the unit where to add the exercise.
					\item 6. Teacher selects the dates where the exercise will be visible.
					\item 7. A list with the four available types of questions (described in System Description 2.1.2.5)is displayed, teacher selects "multiple choice".
					\item 8. Teacher writes down how many questions the exercise will have.
					\item 9. Teacher selects the order of the questions in the exercise (random or not).
					\item 10. Teacher selects the penalty, if any, for each wrong answer.
					\item 11. Teacher selects the value of the exercise in the final mark.
					\item 11. For each question, teacher decides how many statements to show,writes them down and assigns a truth value to each statement.
					\item 12. Teacher will fill a table writing down the points associated to each question.
					\item 12. When all the info is filled in, teacher selects "Save exercise". 
				\end{itemize}
				\item{Extensions}
				\begin{itemize}
					\item Extension 1: one choice test
					\begin{itemize}
						\item 7a. Teacher selects "one choice" exercise.
						\item 11a. For each question, teacher decides how many statements to show, writes them down and \textit{selects one} true statement.
					\end{itemize}
					\item Extension 2: Teacher selects "true/false" exercise.
					\begin{itemize}
						\item 7b. Teacher selects "true/false" exercise.
						\item 11b. For each question, teacher 	\textit{writes down one} statement and assigns a truth value to it.
					\end{itemize}
					\item Extension 3: Teacher selects "open answer" exercise.
					\begin{itemize}
						\item 7c. Teacher selects "open answer" exercise.
						\item 11c. For each question, teacher writes down \textit{one question} and writes down \textit{one answer}.
					\end{itemize}
				
				\end{itemize}
				\item Special Requirements:
				\item Technology and Data Variations List:
				\item Frequency: No concurrency
				\item Open Issues
			\end{itemize}
			
		
		
			\item{Use case: User applies for a course}
		
			
			\begin{itemize}
				\item Primary actor: Student
				\item Stakeholders and goals: Students who want to apply for a new course
				\item Preconditions: To have logged in the application.
				\item Success Guarantee: an email will be sent back with the teacher's answer
				\item Main Success Scenario
				\begin{itemize}
					\item 1. Student selects "Application".
					\item 2. A list of available courses is displayed.
					\item 3. Student selects the course he wants.
					\item 4. An "Are you sure you want to apply for this course?" message is displayed on screen.
					\item 5. Student selects "Yes".
					\item 6. Student selects "Exit application mode".
					\item 7. A notification with the application is sent via e-mail to the teacher.
				\end{itemize}
				\item Extensions
				\begin{itemize}
					\item Extension 1
					\begin{itemize}
						\item 5a. Student selects "No".
						\item 6a. Student selects "Exit application mode".
					\end{itemize}
					
				\end{itemize}
				\item Technology and Data Variations List: If teacher accepts, the Courses Students List is modified.
				\item Frequency: no concurrency, one user at a time.
				
			\end{itemize}
		\end{enumerate}
	\end{enumerate}
	\item{Mockups}
	

\end{enumerate}
\end{document}